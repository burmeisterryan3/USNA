%%%%%%%%%%%%%%%%%%%%%%%%%%%%%%%%%%%%%%%%
% Short Sectioned Assignment
% LaTeX Template
% Version 1.0 (5/5/12)
%
% This template has been downloaded from:
% http://www.LaTeXTemplates.com
%
% Original author:
% Frits Wenneker (http://www.howtotex.com)
%
% License:
% CC BY-NC-SA 3.0 (http://creativecommons.org/licenses/by-nc-sa/3.0/)
%
%%%%%%%%%%%%%%%%%%%%%%%%%%%%%%%%%%%%%%%%%

%----------------------------------------------------------------------------------------
% PACKAGES AND OTHER DOCUMENT CONFIGURATIONS
%----------------------------------------------------------------------------------------

\documentclass[fontsize=12pt]{article}
\usepackage[english]{babel} % English language/hyphenation
\usepackage{amsmath,amsfonts,amsthm} % Math packages

\usepackage{sectsty} % Allows customizing section commands
\allsectionsfont{\centering \normalfont\scshape} % Make all sections centered, the default font and small caps

\usepackage{fancyhdr} % Custom headers and footers
\usepackage[margin=1in]{geometry}
\pagestyle{fancyplain} % Makes all pages in the document conform to the custom headers and footers
\setlength{\headheight}{13.6pt} % Customize the height of the header

\numberwithin{equation}{section} % Number equations within sections (i.e. 1.1, 1.2, 2.1, 2.2 instead of 1, 2, 3, 4)
\numberwithin{figure}{section} % Number figures within sections (i.e. 1.1, 1.2, 2.1, 2.2 instead of 1, 2, 3, 4)
\numberwithin{table}{section} % Number tables within sections (i.e. 1.1, 1.2, 2.1, 2.2 instead of 1, 2, 3, 4)

%\setlength\parindent{0pt} % Removes all indentation from paragraphs - comment this line for an assignment with lots of text

%----------------------------------------------------------------------------------------
% TITLE SECTION
%----------------------------------------------------------------------------------------

\newcommand{\horrule}[1]{\rule{\linewidth}{#1}} % Create horizontal rule command with 1 argument of height

\title{ 
  \normalfont \normalsize 
  \textsc{Randomness and Computation} \\ [25pt] % Your university, school and/or department name(s)
  \horrule{0.5pt} \\[0.4cm] % Thin top horizontal rule
  \huge Problem 13 \\ % The assignment title
  \horrule{2pt} \\[0.5cm] % Thick bottom horizontal rule
}

\author{Ryan Burmeister} % Your name

\date{\normalsize\today} % Today's date or a custom date

\begin{document}

\maketitle % Print the title

The article is about the testing and application of random number generators.
The author, Robert Davies, delves into how to evaluate the effectiveness of
random generators and why anyone should care.  He defines a truly random number
generator as one which can produce a series of bit which are entirely
independent.  The higher the correlation between bits, the worse the generator.




\par He begins by introducing hardware and pseudo-random generators.  He favors
hardware generators as they are more apt to produce statistically independent
bits and more naturally tend to fit the for requirements he gives for random
number generation.  However, he does indicate that pseudo-random number
generators are sufficient in most cases.  The cases in which they are not have
to deal with long term dependicies between data.  In a lottery example, he
demonstrates why hardware, if done correctly, can eliminate the necessity to
verify that the probability numbers drawn by computers are the same as those
which would have been drawn from an urn.  With a pseudo-random number
generator, it would be hard to prove that the probability of the balls drawn
have the same probability of those drawn from a perfectly random scenario, as
with the urn.  The hardware generator allows for the focus on the evaluation of
the other three requirements for computer based draws.

\par The author then goes into his evaluation and analysis of hardware random
number generators.  He began by evaluating the a disk of random number captured
and was able to determine an error in the programming of the individual
capturing the numbers.  I believe the author was attempting to demonstrate the
ease by which the correlation between bits can be driven up due to a
programming error even with an effective hardware generator.  In evaluating the
Canadian, German, and Californian generators, get measure the percentage of 1s
experience in each of the bits while also checking the cross-correlation
between bits, sampling rate, and long term drift and perodicities. He then
delves into determing the how good a random number generator needs to be by
giving using lotteries, statistical simulations, and encryption ase examples.

\par While the paper is interesting, I ended the paper not having a good feel
as to why anyone would need truly random numbers to such a degree as examined
in this article.  For example, in the encryption example, I question ones
ability to predict the key when the range of keys is of sufficient length.
Even if there is a slight correlation between bits, does that necessarily mean
that I generate some algorithm that can generate the correct keys?  With each
of the examples given, it appears the correlation between bits only drives up
probabilistic likelihood by which you could predict some random outcome.
However, how much it improves this likelihood in comparison to other methods I
do not know.  To improve my own understanding, I guess I would need to better
understand the ease by which someone trying to break some encryption scheme
with pseudo-random keys would be able to do so.
\end{document}
