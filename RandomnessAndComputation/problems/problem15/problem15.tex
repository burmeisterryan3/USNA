%%%%%%%%%%%%%%%%%%%%%%%%%%%%%%%%%%%%%%%%
% Short Sectioned Assignment
% LaTeX Template
% Version 1.0 (5/5/12)
%
% This template has been downloaded from:
% http://www.LaTeXTemplates.com
%
% Original author:
% Frits Wenneker (http://www.howtotex.com)
%
% License:
% CC BY-NC-SA 3.0 (http://creativecommons.org/licenses/by-nc-sa/3.0/)
%
%%%%%%%%%%%%%%%%%%%%%%%%%%%%%%%%%%%%%%%%%

%----------------------------------------------------------------------------------------
% PACKAGES AND OTHER DOCUMENT CONFIGURATIONS
%----------------------------------------------------------------------------------------

\documentclass[fontsize=12pt]{article}
\usepackage[english]{babel} % English language/hyphenation
\usepackage{amsmath,amsfonts,amsthm} % Math packages

\usepackage{sectsty} % Allows customizing section commands
\allsectionsfont{\centering \normalfont\scshape} % Make all sections centered, the default font and small caps
\usepackage[margin=1in]{geometry}

\usepackage{fancyhdr} % Custom headers and footers
\pagestyle{fancyplain} % Makes all pages in the document conform to the custom headers and footers
\setlength{\headheight}{13.6pt} % Customize the height of the header

\numberwithin{equation}{section} % Number equations within sections (i.e. 1.1, 1.2, 2.1, 2.2 instead of 1, 2, 3, 4)
\numberwithin{figure}{section} % Number figures within sections (i.e. 1.1, 1.2, 2.1, 2.2 instead of 1, 2, 3, 4)
\numberwithin{table}{section} % Number tables within sections (i.e. 1.1, 1.2, 2.1, 2.2 instead of 1, 2, 3, 4)

%\setlength\parindent{0pt} % Removes all indentation from paragraphs - comment this line for an assignment with lots of text

%----------------------------------------------------------------------------------------
% TITLE SECTION
%----------------------------------------------------------------------------------------

\newcommand{\horrule}[1]{\rule{\linewidth}{#1}} % Create horizontal rule command with 1 argument of height

\title{ 
  \normalfont \normalsize 
  \textsc{Randomness and Computation} \\ [25pt] % Your university, school and/or department name(s)
  \horrule{0.5pt} \\[0.4cm] % Thin top horizontal rule
  \huge Problem 15 \\ % The assignment title
  \horrule{2pt} \\[0.5cm] % Thick bottom horizontal rule
}

\author{Ryan Burmeister} % Your name

\date{\normalsize\today} % Today's date or a custom date

\begin{document}

\maketitle % Print the title

\section{What is Bull Mountain?}
From a security perspective, random numbers are vitale to ensure the generation
of secure keys.  The more random the production of keys is the harder it is for
an advesary to beat the encryption scheme.  There have been methods implemented
in software and hardware, but their are limitations with both.  Software
implementations are inherenetly flawed as a computer is always in a
well-defined state which only changes when processes tell it to.  Hardware
implementations are better as they can rely on the chaotic world around us.
However, they often cannot produce random numbers a rate fast enough to keep up
with the demand of several processes.  Intel set out on a project "Bull
Mountain" to make a random number generator implemented in hardware which could
overcome the typical shortcomins of previous hardware implementations. 
\section{How does it work?}
Like previously mentioned, "Bull Mountain" is a hardware implementation created
to ease the generation of random numbers while still ensuring the quality of
numbers produced (very low bias and correlation). The design is dependent on
the use of an entropy source, just like other hardware implementations.  The
entropy souce used in Bull Mountain is an all digital implementation which
relies on inverters.  In order to capture the random component, two transistors
are added to the circuit between the inverters which forces the state to be the
same for the inverters (without the transistors, if the input to one inverter
was 1 then the output of that inverter was 0 which creates a cycle between the
two inverters).  Temporarily, the inverters are at the same state until random
thermal noise sends the circuit back into one of its two stable states which
produces a single bit which can be used.  To repeat this process, the
transistors are hooked up to a clock which forces the cycle to repeat based off
the clock's period.
\section{What components in a processor are necessary?}
To make all of this possible, additional hardware components are necessary.
The three components are a hardware entropy source (the circuit I described
above), a conditioner, a deterministic random bit generator, and an enhanced
non-determnistic random number generator.  The conditioner groups 256 bits from
the entropy source and combines them using an encryption scheme known as
AES-CBC-MAC.  The output is also 256 bits and is passed to the determnistic
random bit generator (DRBG) to be used as a seed value. This allows for the
random number generated by the conditioner to be used to generate addiitonal
random numbers faster than can be produced in the hardware.  The DRBG decides
when it needs to be reseeded with an upper bound of 1022 sequential random
values used with any one seed.  The enhanced non-deterministic random number
generator offers a way for software based DRBGs to obtain seeds.  As opposed to
producing random numbers directly in hardware, this method allows for the
generation of random numbers in software with a hardware based seed.
\section{How does it compare?}
The first requirement that Intel engineers had was to produce a random number
generator compliant with NIST standards.  Furthermore, they build in methods to
ensure bits were not too biased.  The two methods used are called Online Health
Tests (OHTs) and Built-In Self Tests (BISTs).  OHTs compare bit patterns
against the expected pattern based on the model of the ES and look at the
sample health over many samples to ensure that the samples remain above a
predetermined threshold of health.  In addition to complying with the NIST
standards, engineers also wanted to produce a method which could produce random
numbers at a much faster rate than previous methods.  They were able to obtain
a rate of 3GHz for a ranodm stream of bits.  Furthermore, while other methods
can consume substantial amounts of power, Bull Mountain requires no additional
external power supply to run and is design to function over the wide range of
operating conditions.

\end{document}
