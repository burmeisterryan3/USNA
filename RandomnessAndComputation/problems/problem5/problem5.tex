%%%%%%%%%%%%%%%%%%%%%%%%%%%%%%%%%%%%%%%%
% Short Sectioned Assignment
% LaTeX Template
% Version 1.0 (5/5/12)
%
% This template has been downloaded from:
% http://www.LaTeXTemplates.com
%
% Original author:
% Frits Wenneker (http://www.howtotex.com)
%
% License:
% CC BY-NC-SA 3.0 (http://creativecommons.org/licenses/by-nc-sa/3.0/)
%
%%%%%%%%%%%%%%%%%%%%%%%%%%%%%%%%%%%%%%%%%

%----------------------------------------------------------------------------------------
% PACKAGES AND OTHER DOCUMENT CONFIGURATIONS
%----------------------------------------------------------------------------------------

\documentclass[fontsize=12pt]{article}
\usepackage[english]{babel} % English language/hyphenation
\usepackage{amsmath,amsfonts,amsthm} % Math packages

\usepackage{sectsty} % Allows customizing section commands
\allsectionsfont{\centering \normalfont\scshape} % Make all sections centered, the default font and small caps

\usepackage{fancyhdr} % Custom headers and footers
\pagestyle{fancyplain} % Makes all pages in the document conform to the custom headers and footers
\setlength{\headheight}{13.6pt} % Customize the height of the header

\numberwithin{equation}{section} % Number equations within sections (i.e. 1.1, 1.2, 2.1, 2.2 instead of 1, 2, 3, 4)
\numberwithin{figure}{section} % Number figures within sections (i.e. 1.1, 1.2, 2.1, 2.2 instead of 1, 2, 3, 4)
\numberwithin{table}{section} % Number tables within sections (i.e. 1.1, 1.2, 2.1, 2.2 instead of 1, 2, 3, 4)

%\setlength\parindent{0pt} % Removes all indentation from paragraphs - comment this line for an assignment with lots of text

%----------------------------------------------------------------------------------------
% TITLE SECTION
%----------------------------------------------------------------------------------------

\newcommand{\horrule}[1]{\rule{\linewidth}{#1}} % Create horizontal rule command with 1 argument of height
\title{ 
  \normalfont \normalsize 
  \textsc{Randomness and Computation} \\ [25pt] % Your university, school and/or department name(s)
  \horrule{0.5pt} \\[0.4cm] % Thin top horizontal rule
  \huge Problem 5 \\ % The assignment title
  \horrule{2pt} \\[0.5cm] % Thick bottom horizontal rule
}

\author{Ryan Burmeister} % Your name

\date{\normalsize\today} % Today's date or a custom date

\begin{document}

\maketitle % Print the title
In my previous attempt, I detailed the math as to why I would not have played the
powerball.  However, if I had to pick a strategy to play the lottery, I would try
and enter with a group.  With this strategy, I increase my odds of winning without
having to substantially increasing my own contribution.  Although my odds may not
be significantly increased, my odds are better.  The split in the payout seems like
a small price to pay to win anything at all.  I don't think anyone goes into the
lottery expecting to win, but I do believe anyone, should they choose to play, would
want to increase their odds of winning in any way possible.

\par If I were someone who played the lottery often, I would select a set of
numbers and always play the same numbers.  In a single play, picking any set of
numbers would not increase my odds, but over the course of many plays I should
expect combinations of numbers not seen would eventually show up (nevermind
that it may take millions of year for it to show up).  Ultimately, if you want
to win the lottery, you should simply play.  Your odds are never increase in a
more significant manner than when you buy your first ticket.

\par In terms of the math I had previously done, playing at a time when the
lottery is the highest it has every been makes the most sense.  Your average
expected payout will never be higher than it is in the scenario.

\par In terms of the math I had previously done, playing at a time when the
lottery is the highest it has every been makes the most sense.  Your average
expected payout will never be higher than it is in the scenario.  Therefore, I
would apply my strategy taken in the first paragraph (group purchase) to this
scenario in hopes that I may win due to my increased odds, no matter how
(in)significant they may be.
\end{document}
