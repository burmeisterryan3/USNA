%%%%%%%%%%%%%%%%%%%%%%%%%%%%%%%%%%%%%%%%
% Short Sectioned Assignment
% LaTeX Template
% Version 1.0 (5/5/12)
%
% This template has been downloaded from:
% http://www.LaTeXTemplates.com
%
% Original author:
% Frits Wenneker (http://www.howtotex.com)
%
% License:
% CC BY-NC-SA 3.0 (http://creativecommons.org/licenses/by-nc-sa/3.0/)
%
%%%%%%%%%%%%%%%%%%%%%%%%%%%%%%%%%%%%%%%%%

%----------------------------------------------------------------------------------------
% PACKAGES AND OTHER DOCUMENT CONFIGURATIONS
%----------------------------------------------------------------------------------------

\documentclass[fontsize=12pt]{article}
\usepackage[english]{babel} % English language/hyphenation
\usepackage{amsmath,amsfonts,amsthm} % Math packages

\usepackage{sectsty} % Allows customizing section commands
\allsectionsfont{\centering \normalfont\scshape} % Make all sections centered, the default font and small caps

\usepackage{fancyhdr} % Custom headers and footers
\pagestyle{fancyplain} % Makes all pages in the document conform to the custom headers and footers
\setlength{\headheight}{13.6pt} % Customize the height of the header

\numberwithin{equation}{section} % Number equations within sections (i.e. 1.1, 1.2, 2.1, 2.2 instead of 1, 2, 3, 4)
\numberwithin{figure}{section} % Number figures within sections (i.e. 1.1, 1.2, 2.1, 2.2 instead of 1, 2, 3, 4)
\numberwithin{table}{section} % Number tables within sections (i.e. 1.1, 1.2, 2.1, 2.2 instead of 1, 2, 3, 4)

%\setlength\parindent{0pt} % Removes all indentation from paragraphs - comment this line for an assignment with lots of text

%----------------------------------------------------------------------------------------
% TITLE SECTION
%----------------------------------------------------------------------------------------

\newcommand{\horrule}[1]{\rule{\linewidth}{#1}} % Create horizontal rule command with 1 argument of height

\title{ 
  \normalfont \normalsize 
  \textsc{Randomness and Computation} \\ [25pt] % Your university, school and/or department name(s)
  \horrule{0.5pt} \\[0.4cm] % Thin top horizontal rule
  \huge Problem 29 \\ % The assignment title
  \horrule{2pt} \\[0.5cm] % Thick bottom horizontal rule
}

\author{Ryan Burmeister} % Your name

\date{\normalsize\today} % Today's date or a custom date

\begin{document}
\maketitle % Print the title

Python:
Python random's class implements the Mersenne Twister as its PRNG.  The period
of the of PRNG is $2^{19937-1}$, and the default seed is set by using the
current system time which is done when the module is first imported.  If a
random source is available, then that source is used to supply seeds.
Therefore, A "true" entropy source is only used if provided be the OS. 

Java:
Java's Java.util.Random implements a linear congruential formula with
48-bit seeds.  The period is only $2^48$.  To generate a random number,
the documentation states that it sets the seed value to one "likely to
be distince from any other invocation of this constructor."  After further
digging, the code consists of a call to seedUniqueifier().  This constructor
follows a method from "Table of Linear Congruential Generaors of Different
Sizes and Good Lattice Structure (1999, L'Ecuyer)" to generate its seed.
The iniital value is set to $8682522807148012$, and every subsequent value
is computed by multiplying the current seed value by $1817783497276652981$ and
xor-ing that number with the current time in nanoseconds.

\end{document}
