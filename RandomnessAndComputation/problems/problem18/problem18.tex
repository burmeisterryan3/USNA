%%%%%%%%%%%%%%%%%%%%%%%%%%%%%%%%%%%%%%%%
% Short Sectioned Assignment
% LaTeX Template
% Version 1.0 (5/5/12)
%
% This template has been downloaded from:
% http://www.LaTeXTemplates.com
%
% Original author:
% Frits Wenneker (http://www.howtotex.com)
%
% License:
% CC BY-NC-SA 3.0 (http://creativecommons.org/licenses/by-nc-sa/3.0/)
%
%%%%%%%%%%%%%%%%%%%%%%%%%%%%%%%%%%%%%%%%%

%----------------------------------------------------------------------------------------
% PACKAGES AND OTHER DOCUMENT CONFIGURATIONS
%----------------------------------------------------------------------------------------

\documentclass[fontsize=12pt]{article}
\usepackage[english]{babel} % English language/hyphenation
\usepackage{amsmath,amsfonts,amsthm} % Math packages

\usepackage{sectsty} % Allows customizing section commands
\allsectionsfont{\centering \normalfont\scshape} % Make all sections centered, the default font and small caps

\usepackage{fancyhdr} % Custom headers and footers
\pagestyle{fancyplain} % Makes all pages in the document conform to the custom headers and footers
\setlength{\headheight}{13.6pt} % Customize the height of the header

\numberwithin{equation}{section} % Number equations within sections (i.e. 1.1, 1.2, 2.1, 2.2 instead of 1, 2, 3, 4)
\numberwithin{figure}{section} % Number figures within sections (i.e. 1.1, 1.2, 2.1, 2.2 instead of 1, 2, 3, 4)
\numberwithin{table}{section} % Number tables within sections (i.e. 1.1, 1.2, 2.1, 2.2 instead of 1, 2, 3, 4)

%\setlength\parindent{0pt} % Removes all indentation from paragraphs - comment this line for an assignment with lots of text

%----------------------------------------------------------------------------------------
% TITLE SECTION
%----------------------------------------------------------------------------------------

\newcommand{\horrule}[1]{\rule{\linewidth}{#1}} % Create horizontal rule command with 1 argument of height

\title{ 
  \normalfont \normalsize 
  \textsc{Randomness and Computation} \\ [25pt] % Your university, school and/or department name(s)
  \horrule{0.5pt} \\[0.4cm] % Thin top horizontal rule
  \huge Problem 18 \\ % The assignment title
  \horrule{2pt} \\[0.5cm] % Thick bottom horizontal rule
}

\author{Ryan Burmeister} % Your name

\date{\normalsize\today} % Today's date or a custom date

\begin{document}

\maketitle % Print the title Shannon entropy is defined as,

$\sum_{i=1}^{n}p_ilg(\frac{1}{p_i})$ where $n$ events are possible and each
event has probability $p_i$.  If assume a biased coing scenario with the
probability of the coin landing on heads as $p$, then the probability of the
coin landing on tails is $1-p$. If we expand the right side of the Shannon
entropy equation with the biased coin scenario, you get
$-plg(p)+(1-p)lg(\frac{1}{1-p})$.  In the unbiased coin scenario ($p=.5$), the
shannon entropy is $1$.  To calculate the number of flips necessary for a
biased coin to obtain the same shannon entropy as an unbiased coin, one can
solve for $p$.  In the case where $p=.75$, the shannon entropy is $\approx
0.811278$.  Therefore, at least two flips are necessary to obtain one bit of
information.  In the more general sense, the least value of $n$ for which the
following equation holds true give the number of flips required for a biased
coin to give the same information as an unbiased one,
$-plg(p)+(1-p)lg(\frac{1}{1-p}) \geq \frac{1}{n}$.


\end{document}
