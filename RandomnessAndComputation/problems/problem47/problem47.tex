%%%%%%%%%%%%%%%%%%%%%%%%%%%%%%%%%%%%%%%%
% Short Sectioned Assignment
% LaTeX Template
% Version 1.0 (5/5/12)
%
% This template has been downloaded from:
% http://www.LaTeXTemplates.com
%
% Original author:
% Frits Wenneker (http://www.howtotex.com)
%
% License:
% CC BY-NC-SA 3.0 (http://creativecommons.org/licenses/by-nc-sa/3.0/)
%
%%%%%%%%%%%%%%%%%%%%%%%%%%%%%%%%%%%%%%%%%

%----------------------------------------------------------------------------------------
% PACKAGES AND OTHER DOCUMENT CONFIGURATIONS
%----------------------------------------------------------------------------------------

\documentclass[fontsize=12pt]{article}
\usepackage[english]{babel} % English language/hyphenation
\usepackage{amsmath,amsfonts,amsthm} % Math packages

\usepackage{sectsty} % Allows customizing section commands
\allsectionsfont{\centering \normalfont\scshape} % Make all sections centered, the default font and small caps

\usepackage{fancyhdr} % Custom headers and footers
\pagestyle{fancyplain} % Makes all pages in the document conform to the custom headers and footers
\setlength{\headheight}{13.6pt} % Customize the height of the header

\numberwithin{equation}{section} % Number equations within sections (i.e. 1.1, 1.2, 2.1, 2.2 instead of 1, 2, 3, 4)
\numberwithin{figure}{section} % Number figures within sections (i.e. 1.1, 1.2, 2.1, 2.2 instead of 1, 2, 3, 4)
\numberwithin{table}{section} % Number tables within sections (i.e. 1.1, 1.2, 2.1, 2.2 instead of 1, 2, 3, 4)

%\setlength\parindent{0pt} % Removes all indentation from paragraphs - comment this line for an assignment with lots of text

%----------------------------------------------------------------------------------------
% TITLE SECTION
%----------------------------------------------------------------------------------------

\newcommand{\horrule}[1]{\rule{\linewidth}{#1}} % Create horizontal rule command with 1 argument of height

\title{ 
  \normalfont \normalsize 
  \textsc{Randomness and Computation} \\ [25pt] % Your university, school and/or department name(s)
  \horrule{0.5pt} \\[0.4cm] % Thin top horizontal rule
  \huge Problem 47 \\ % The assignment title
  \horrule{2pt} \\[0.5cm] % Thick bottom horizontal rule
}

\author{Ryan Burmeister} % Your name

\date{\normalsize\today} % Today's date or a custom date

\begin{document}

\maketitle % Print the title
The Flajolet\_Martin algorithm is used to approximate the number of unique
elements in a set.  Using the algorithm also ensure that only a single pass
over the set is required and that the amount of space required is logarithmic
compared to the maximum possible number of distinct elements.

The algorithm works by initializing a vector of length $L$ to all $0$s.  We
then assume that we have a hash function which maps an input $x$ to a value in
the range $\left[0,2^L-1\right]$.  We then determine the first index for which
the bit is $1$ in the hash.  This index is then set to $1$ in our vector of
length $L$.  $R$ is then defined to be the smalles index in our vector which is
$0$.  By performing the computation $2^R/\phi$, where $\phi\approx0.77351$, we
can estimate the cardinality of our vector.

From an intuitive standpoint, this algorithm seems to make a lot of sense.  The
probability that each hash is even or odd is $\frac{1}{2}$.  Therefore, the
first bit in our vector will be set to $1$ approximately $50\%$ of the time.
The chances that more significant bits are set to $1$ occur with probability
$\frac{1}{2^n}$ given that our hash function uniformly distributes the output. \\

Example: \\ \begin{center}
$13 = 1101_2 \rightarrow 0$, $0^{th}$ bit in vector is set to 1 \\
$8 = 1000_2 \rightarrow 3$, $3^{rd}$ bit in vector is set to 1 \\
$12 = 1100_2 \rightarrow 2$, $2^{nd}$ bit in vector is set to 1 \\
$9 = 1001_2 \rightarrow 1$, $0^{th}$ bit in vector is set to 1 \\
$10 = 1010_2 \rightarrow 1$, $1^{st}$ bit in vector is set to 1 \\
\end{center}
vector = 01111 \\
The esitmate of the cardinality of the set from which the hashes were drawn is
$2^4/0.77351 \approx 20.57$.

Sources:
Wikipedia - Flajolet-Martin Algorithm \\
Flajolet, Philippe; Martin, G. Nigel (1985). "Probabilistic counting algorithms for data base applications"
\end{document}
