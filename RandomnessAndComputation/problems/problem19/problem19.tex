%%%%%%%%%%%%%%%%%%%%%%%%%%%%%%%%%%%%%%%%
% Short Sectioned Assignment
% LaTeX Template
% Version 1.0 (5/5/12)
%
% This template has been downloaded from:
% http://www.LaTeXTemplates.com
%
% Original author:
% Frits Wenneker (http://www.howtotex.com)
%
% License:
% CC BY-NC-SA 3.0 (http://creativecommons.org/licenses/by-nc-sa/3.0/)
%
%%%%%%%%%%%%%%%%%%%%%%%%%%%%%%%%%%%%%%%%%

%----------------------------------------------------------------------------------------
% PACKAGES AND OTHER DOCUMENT CONFIGURATIONS
%----------------------------------------------------------------------------------------

\documentclass[fontsize=12pt]{article}
\usepackage[english]{babel} % English language/hyphenation
\usepackage{amsmath,amsfonts,amsthm} % Math packages
\usepackage{algorithm}
\usepackage{algpseudocode}

\usepackage{sectsty} % Allows customizing section commands
\allsectionsfont{\centering \normalfont\scshape} % Make all sections centered, the default font and small caps

\usepackage{fancyhdr} % Custom headers and footers
\pagestyle{fancyplain} % Makes all pages in the document conform to the custom headers and footers
\setlength{\headheight}{13.6pt} % Customize the height of the header

\numberwithin{equation}{section} % Number equations within sections (i.e. 1.1, 1.2, 2.1, 2.2 instead of 1, 2, 3, 4)
\numberwithin{figure}{section} % Number figures within sections (i.e. 1.1, 1.2, 2.1, 2.2 instead of 1, 2, 3, 4)
\numberwithin{table}{section} % Number tables within sections (i.e. 1.1, 1.2, 2.1, 2.2 instead of 1, 2, 3, 4)

%\setlength\parindent{0pt} % Removes all indentation from paragraphs - comment this line for an assignment with lots of text

%----------------------------------------------------------------------------------------
% TITLE SECTION
%----------------------------------------------------------------------------------------

\newcommand{\horrule}[1]{\rule{\linewidth}{#1}} % Create horizontal rule command with 1 argument of height

\title{ 
  \normalfont \normalsize 
  \textsc{Randomness and Computation} \\ [25pt] % Your university, school and/or department name(s)
  \horrule{0.5pt} \\[0.4cm] % Thin top horizontal rule
  \huge Problem 19 \\ % The assignment title
  \horrule{2pt} \\[0.5cm] % Thick bottom horizontal rule
}

\author{Ryan Burmeister} % Your name

\date{\normalsize\today} % Today's date or a custom date

\begin{document}

\maketitle % Print the title
Entropy for a fair 3-sided die: \begin{center} $-\sum_{i=1}^{3}p_ilg(p_i) =
-\frac{1}{3}lg(\frac{1}{3})*3 = -lg(\frac{1}{3}) = lg(3) \approx 1.585$
\end{center}
Entropy for a weighted 6-sided die: \begin{center}
$-\sum_{i=1}^{6}p_ilg(p_i) =
\frac{1}{6}lg(6)+\frac{1}{9}lg(9)*3+\frac{1}{4}lg(4)*2 =
\frac{1}{6}lg(6)+\frac{1}{3}lg(9)+1\approx2.487$ \end{center} 

An algorithm for simulating the fair 3-sided die with the weighted
6-sided die is below(\ref{alg: algorithm}).

\begin{algorithm}
  \caption{algorithm}\label{alg: algorithm}
  \begin{algorithmic}[1]
    \While {true:}
      \State roll die
      \If {roll == 2 or roll == 3 or roll == 4:}
        \State print 'A'
      \Else
        \State roll again
        \If {roll == 5 or roll == 6:}
          \State print 'B'
        \Else
          \State print 'C'
        \EndIf
      \EndIf
    \EndWhile
 \end{algorithmic} 
\end{algorithm}
    
This algorithm is correct because we can treat the two rolls as independent
events.  The probability of rollina an A is $3*\frac{1}{9} = \frac{1}{3}$.
The probability of rolling either B or C will each have a chance of
occuring $\frac{2}{3}$ of the time (i.e. Prob(B or C) = Prob({\=A}).  The
probability of B is then $\frac{2}{3}*(\frac{1}{4}+\frac{1}{4}) =
\frac{1}{3}$.  The probability of C can be calculated in a similar fashion,
$\frac{2}{3} * (\frac{1}{6} + 3*\frac{1}{9}) = \frac{1}{3}$.

The expected number of dice rolls can be computed through the expected value.
$E[X] = \sum_{i=1}^{n}p_iv_i = \frac{1}{3}*1 + \frac{2}{3}*2 =
\frac{1}{3}+\frac{4}{3} = \frac{5}{3}$.  The entropy of the the algorithm per
output is then: $\frac{5}{3}(\frac{1}{6}lg(6)+\frac{1}{3}lg(9)+1) \approx
4.146$.  The entropy is then compared to the scenario with our fair 3-sided die
to compute the entropy loss.
$\frac{\frac{5}{3}(\frac{1}{6}lg(6)+\frac{1}{3}lg(9)+1)-lg(3)}{\frac{5}{3}(\frac{1}{6}lg(6)+\frac{1}{3}(9)+1)}
\approx \frac{4.146-1.585}{4.146} \approx 0.6177$.

\end{document}
