%%%%%%%%%%%%%%%%%%%%%%%%%%%%%%%%%%%%%%%%
% Short Sectioned Assignment
% LaTeX Template
% Version 1.0 (5/5/12)
%
% This template has been downloaded from:
% http://www.LaTeXTemplates.com
%
% Original author:
% Frits Wenneker (http://www.howtotex.com)
%
% License:
% CC BY-NC-SA 3.0 (http://creativecommons.org/licenses/by-nc-sa/3.0/)
%
%%%%%%%%%%%%%%%%%%%%%%%%%%%%%%%%%%%%%%%%%

%----------------------------------------------------------------------------------------
% PACKAGES AND OTHER DOCUMENT CONFIGURATIONS
%----------------------------------------------------------------------------------------

\documentclass[fontsize=12pt]{article}
\usepackage[english]{babel} % English language/hyphenation
\usepackage{amsmath,amsfonts,amsthm} % Math packages

\usepackage{sectsty} % Allows customizing section commands
\allsectionsfont{\centering \normalfont\scshape} % Make all sections centered, the default font and small caps

\usepackage{fancyhdr} % Custom headers and footers
\pagestyle{fancyplain} % Makes all pages in the document conform to the custom headers and footers
\setlength{\headheight}{13.6pt} % Customize the height of the header

\numberwithin{equation}{section} % Number equations within sections (i.e. 1.1, 1.2, 2.1, 2.2 instead of 1, 2, 3, 4)
\numberwithin{figure}{section} % Number figures within sections (i.e. 1.1, 1.2, 2.1, 2.2 instead of 1, 2, 3, 4)
\numberwithin{table}{section} % Number tables within sections (i.e. 1.1, 1.2, 2.1, 2.2 instead of 1, 2, 3, 4)

%\setlength\parindent{0pt} % Removes all indentation from paragraphs - comment this line for an assignment with lots of text

%----------------------------------------------------------------------------------------
% TITLE SECTION
%----------------------------------------------------------------------------------------

\newcommand{\horrule}[1]{\rule{\linewidth}{#1}} % Create horizontal rule command with 1 argument of height

\title{ 
  \normalfont \normalsize 
  \textsc{Randomness and Computation} \\ [25pt] % Your university, school and/or department name(s)
  \horrule{0.5pt} \\[0.4cm] % Thin top horizontal rule
  \huge Problem 62 \\ % The assignment title
  \horrule{2pt} \\[0.5cm] % Thick bottom horizontal rule
}

\author{Ryan Burmeister} % Your name

\date{\normalsize\today} % Today's date or a custom date

\begin{document}

\maketitle % Print the title
\section{Fastest Search}
The fastest search time would be between the rBST and Treap as a Skip List
would require the traversal over links with variable points in memory.  The
Treap would seem to have the fastest search time as it seems to be less biased
towards the order of its inputs.  A rBST would seem to have a greater chance to
look closer like a degenerate tree if all of its inputs were in ascending or
descening order.  For that reason, a Treap would have the fastest searh time.

\section{Fastest Insertion}
Treaps would have the fastest insertion time as it simply requires the addition
of an item to an array with some bubbling up as necessary.  A Skip List
requires the maniuplation of pointers and the creation of a node with a Tower
of a particular height based on the number of flips.  rBSTs require the
restructuing of the tree based on where the new node is inserted.

\section{Fastest Deletion}
Treaps would have the fastest deletions.  Skip Lists would require the changing
of a number of pointers equivalent to the height.  A rBST would again require the
restructuring of the subtree in order to determine which nodes should be the roots
of each subtree.  Treaps would simply require moving elements in right diagonal of
the subtree up the tree.

\section{Simplest to Implement}
A Skip List would be the easiest to implement as all that it would require is
the creation of Towers (arrays of pointers) and the changing of pointers for
insertion and deletion.  Each of the tree methods require the user to correctly
swap elements in an array with the insertion or deletion of nodes.

\section{Smallest Memory Footprint}
rBSTs would have the smallest footprint of the three.  Both rBSTs and Treaps
only store their data and a couple of pointers while Skip Lists store arrays
based on the height of each tower.  Furthermore, rBSTs just have to store the
data while Treaps also store a value for the priority of each node.

\end{document}
